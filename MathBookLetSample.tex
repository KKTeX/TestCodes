%%%%%%%%%%%%%%%%%%%%%%%%%%%%%%%%%%%%%%%%%%%%%%%%%%%%%%%%%%%%%%%%%%%%%%%%%%%%%%%%%%%%%%%%%
\documentclass[luatex,fontsize=8pt,paper=b5,twoside,report]{jlreq}%今後はこっちメインで
\usepackage[top=16truemm,bottom=14truemm,left=12truemm,right=12truemm]{geometry}%(電子書籍はこっち)
% \usepackage[top=16truemm,bottom=14truemm,outer=15truemm,inner=18truemm]{geometry}%(紙の書籍はこっち)
%%%%%%%%%%%%%%%%%%%%%%%%%%%%%%%%%%%%%%%%%%%%%%%%%%%%%%%%%%%%%%%%%%%%%%%%%%%%%%%%%%%%%%%%%


%パッケージ関連%%%%%%%%%%%%%%%%%%%%%%%%%%%%%%%%%%%%%%%%%%%%%%%%%%%%%%%%%%%%%%%%%%%%%%%%%%
\usepackage{/Users/majinkuu/Documents/TeXDevelop/styles2/packages-v1.0}
\usepackage{/Users/majinkuu/Documents/TeXDevelop/styles2/colors-v1.0}
\usepackage{/Users/majinkuu/Documents/TeXDevelop/styles2/luatex-hiragino-v1.3}
\usepackage{/Users/majinkuu/Documents/TeXDevelop/styles2/pre-KKawaguchi-v2.0}
\usepackage{/Users/majinkuu/Documents/TeXDevelop/styles2/stfile-of-KKwaguchi-v1.6}
\usepackage{/Users/majinkuu/Documents/TeXDevelop/styles2/sectioncustomize2}
% \usepackage[usetype1]{/Users/majinkuu/Documents/TeXDevelop/styles2/使わないstyファイル/myuline--}
% \usepackage{background}
%%%%%%%%%%%%%%%%%%%%%%%%%%%%%%%%%%%%%%%%%%%%%%%%%%%%%%%%%%%%%%%%%%%%%%%%%%%%%%%%%%%%%%%%%

\pagestyle{fancy}
\renewcommand{\headrule}{}
\fancyhead{}

\fancyhead[ER]{\raisebox{.7mm}{\mgfamily\footnotesize 数学実戦講座 \ajRoman{1}/\ajRoman{2} 第{\large XX}回 「指数・対数関数」}}
\fancyhead[OL]{\raisebox{.7mm}{\mgfamily\footnotesize 2025年度 高2数学 XXクラス}}
\chead{\vskip5mm \hspace{-4pt} \dotfill \hspace{-1pt}}

% \fancyfoot[ER]{\Lhashira{\hashiraextraL}}
% \fancyfoot[OL]{\Rhashira{\hashiraextraR}}

\cfoot{\vskip-30pt  \hspace{-4pt} \dotfill \hspace{-1pt}}

\tocloftpagestyle{fancy}
%----------------------------------------------------------------------------------------

\begin{document}
\begin{ascolorbox4}[静岡大]{例題1}
   不等式$\di \log_x{y} + 2\log_{\frac{1}{2}}{x} - \log_{\frac{1}{2}}{y} - 2 > 0$の表す領域を図示せよ。
\end{ascolorbox4}
\begin{multicols*}{2}%
\noindent\DataItemize{Data}{%
  ● テーマ :\fft{対数不等式と領域}\\
  ● 難易度 :$A^+$(発送:$A$, 作業$B^-$)\\
  ● 目標時間:\fft{20分}(計算量:$A^+$)
}

\ascboxZ{対数不等式}
\tri やるべきことは明確で、

\begin{ptbs}{KEY}[$\log$を見かけたら]
   初手で\fft{底条件・真数条件}をケア
\end{ptbs}

にしたがって、

\[x, y が正の数, x \neq 1\]

をまず考えてあげましょう。

\tri その後は\stength{対数不等式}の基本に忠実に、

\begin{ptbs}{KEY}[$\log$の計算の基本方針]
   底を統一(なるべく1より大きいものに)
\end{ptbs}

に従います。今回の統一先としては、

\begin{mydec1}\centering
  \maru{1} $x$ \hfill \maru{2} $\di \frac{1}{2}$ \hfill \maru{3} $2$
\end{mydec1}

などが考えられるでしょう。\\
\tri \maru{1}の$x$に統一しようとすると、

\begin{align*}
  \text{(与式)} &\Leftrightarrow \log_x{y} + 2 \frac{1}{\log_x{\frac{1}{2}}} - \frac{\log_x{y}}{\log_x{\frac{1}{2}}} - 2 > 0\\
  &\Leftrightarrow  \log_x{y} - 2 \frac{1}{\log_x{2}} + \frac{\log_x{y}}{\log_x{2}} - 2 > 0
\end{align*}

となり、$\log_x{y}$と$\log_x{2}$が林立しますが、もともと$x, y$について対称性の高い式であり、また最終的に聞かれているのも$xy$平面の話なので、ここから式変形するのは少々やり辛いです。

\begin{mydec4}
 今回底を$x$に統一するとうまくいかない理由は、$\log_x{y}$で全てを表すことができないからです。逆に言えばそうでない問題においては有用です。➡︎\midashi{補足}
\end{mydec4}

\tri \maru{2} は$(底)< 1$となり不等式を解く過程で\snamift{不等号がひっくり返る}ため面倒です。よって避けるべきで、\maru{3}に統一してあげましょう。

\begin{ascolorbox10}{底を$2$に統一}
  \begin{align*}
  \text{(与式)} \quad 
  &\Leftrightarrow \frac{\log_2 y}{\log_2 x} 
  + 2 \cdot \frac{\log_2 x}{\log_2 \frac{1}{2}} 
  - \frac{\log_2 y}{\log_2 \frac{1}{2}} - 2 > 0 \\
  &\Leftrightarrow \frac{\log_2 y}{\log_2 x} 
  - 2 \log_2 x 
  + \log_2 y - 2 > 0
  \end{align*}
\end{ascolorbox10}

\ascboxZ{カタマリとみる}
\tri あとは\fbox{カタマリ}を考えるのみです。

\begin{ptbs}{KEY}[\fbox{カタマリ}とみる]
   同じ式が複数登場していたり、複雑な式の部分が存在するとき\\
  ➡︎\fbox{カタマリ}と見て文字でおく!
\end{ptbs}

後の変形は板書に譲るとしましょう。

\appendixtitle{補足資料}
\tri 結局、全体の流れは

\begin{ptbs}{KEY}[対数方程式・不等式(指数方程式・不等式も同様)]
  \begin{description}
    \item[Step1.] 真数条件・底条件をチェック
    \item[Step2.] 底の統一(なるべく1より大きいものに)  
    \item[Step3.] $\log = \log$ と両辺に $\log$ 1つずつの形に変形
    \item[Step4.] 真数の比較 
  \end{description}
  \maru{\textmg{注}} 底が1より小さいと不等号の向きが逆になるので注意\\
  \maru{\textmg{注}} 真数にルートや分数が出ないようにするために
  \begin{itemize}
    \item[\maru{1}] 分数の計数は払う
    \item[\maru{2}] $\log$ の引き算は移項して足し算に直す 
  \end{itemize}
\end{ptbs}
に終始します。\\
 典型問題ですから、一度も手を止めることなく正解を得てください。
\end{multicols*}

\vspace*{.5mm}\minititle{板書用スペース}\vspace*{.6mm}

\notefill

\begin{ascolorbox14}{\grayhighlight{\textmg{予習で気づかなかったポイント・知らなかったポイント}}}
  \vspace*{6em}
\end{ascolorbox14}
\end{document}