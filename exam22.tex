%%%%%%%%%%%%%%%%%%%%%%%%%%%%%%%%%%%%%%%%%%%%%%%%%%%%%%%%%%%%%%%%%%%%%%%%%%%%%%%%%%%%%%%%%
\documentclass[luatex,fontsize=8pt,paper=b5,twoside,report]{jlreq}%今後はこっちメインで
\usepackage[top=12truemm,bottom=14truemm,left=12truemm,right=12truemm]{geometry}%(電子書籍はこっち)
% \usepackage[top=16truemm,bottom=14truemm,outer=15truemm,inner=18truemm]{geometry}%(紙の書籍はこっち)
%%%%%%%%%%%%%%%%%%%%%%%%%%%%%%%%%%%%%%%%%%%%%%%%%%%%%%%%%%%%%%%%%%%%%%%%%%%%%%%%%%%%%%%%%


%パッケージ関連%%%%%%%%%%%%%%%%%%%%%%%%%%%%%%%%%%%%%%%%%%%%%%%%%%%%%%%%%%%%%%%%%%%%%%%%%%
\usepackage{/Users/majinkuu/Documents/TeXDevelop/styles2/packages-v1.0}
\usepackage{/Users/majinkuu/Documents/TeXDevelop/styles2/colors-v1.0}
\usepackage{/Users/majinkuu/Documents/TeXDevelop/styles2/luatex-hiragino-v1.3}
\usepackage{/Users/majinkuu/Documents/TeXDevelop/styles2/pre-KKawaguchi-v2.0}
\usepackage{/Users/majinkuu/Documents/TeXDevelop/styles2/stfile-of-KKwaguchi-v1.6}
\usepackage{/Users/majinkuu/Documents/TeXDevelop/styles2/sectioncustomize2}
% \usepackage[usetype1]{/Users/majinkuu/Documents/TeXDevelop/styles2/使わないstyファイル/myuline--}
% \usepackage{background}
%%%%%%%%%%%%%%%%%%%%%%%%%%%%%%%%%%%%%%%%%%%%%%%%%%%%%%%%%%%%%%%%%%%%%%%%%%%%%%%%%%%%%%%%%

%ヘッダー関連----------------------------------------------------------------------------
\pagestyle{fancy}
\renewcommand{\headrule}{}\fancyhead{}

\fancyhead[L]{\raisebox{.7mm}{\mgfamily \footnotesize 東京大学}}
\fancyhead[R]{\raisebox{.7mm}{\mgfamily \footnotesize 力学A期末試験問題}}
\chead{\vskip5mm \hspace{-4pt} \dotfill \hspace{-1pt}}

% \fancyfoot[ER]{\Lhashira{\hashiraextraL}}
% \fancyfoot[OL]{\Rhashira{\hashiraextraR}}

\cfoot{\vskip-30pt  \hspace{-4pt} \dotfill \hspace{-1pt}}
%----------------------------------------------------------------------------------------

\begin{document}

\begin{ascolorbox4}[\importances[重要度]{MAX}]{試験問題1}
   重力加速度の大きさ $g$ の一様な重力のもとで,質量 $m$ の質点を,時刻 $t = 0$ に速さ $v_{0}$ で鉛直上向き
  に投射する.質点には速度に比例する空気抵抗(比例定数を$ k (> 0) $ とする)がはたらくとする.時刻 $t$ における質点の速度(鉛直上向きを正とする)を $v(t)$ と表す.
  \begin{mondaiA}
    \item  質点の運動方程式を書け.
    \item  質点の高さが最大となる時刻を求めよ.
    \item  初期位置から測った質点の最高到達地点の高さを求めよ.
    \item  十分時間が経った後,質点はどのような運動をしているか,説明せよ.
  \end{mondaiA}
\end{ascolorbox4}

\begin{multicols*}{2}
\midashi{解答}

\begin{mondaiA}
\item  EOM より 

\[
m\dot{v}(t) = -mg -kv(t) \numbering{\maru{1}}
\]

\noindent であり,\maru{1}$\numbering{(答)}$

\maru{1} を解くと,

\begin{align*}
  m\frac{dv}{dt} &= -k\left(v + \frac{mg}{k}\right) \\
  \frac{dv}{v+\frac{mg}{k}} &= -\frac{k}{m}\,dt\\
  \log{\left|v+\frac{mg}{k}\right|} &= -\frac{k}{m}t + C \;(C=const.)
\end{align*}

終端速度を $v_{\infty}$ とすると,力の釣り合いより

\[
0 = kv_{\infty} + mg \quad \therefore \; v_{\infty} = -\frac{mg}{k}
\]

\noindent よって $\di -\frac{mg}{k} < v(t) \leqq v_{0}$ だから,
$\di \left| v + \frac{mg}{k} \right| = v + \frac{mg}{k}$

\[
\therefore v + \frac{mg}{k} = C_{0}e^{-\frac{k}{m}t} \; (\text{新たに定数$C_{0}$を置いた。})
\]

$t = 0$ で $v = v_{0}$ なので,

\begin{align*}
C_{0} &= v_{0} + \frac{mg}{k}\\
\therefore \quad v &= \left(v_{0} + \frac{mg}{k}\right)e^{-\frac{k}{m}t} - \frac{mg}{k}
\end{align*}

\item  質点の高さが最大の時 $v = 0$ なので,

\begin{align*}
  e^{-\frac{k}{m}t} &= \frac{kv_{0} + mg}{mg}\\
  t &= \frac{k}{m}\log{v+\frac{mg}{k}}\numbering{(答)} \; (=t_{0}とする) 
\end{align*}

\begin{mydec4}
\centering 2018-1(3)と同じ。
\end{mydec4}

\item \begin{align*}
  \Delta x &= \int_{o}^{t_{0}} v(t) \; dt\\
  &= \left[-\frac{mg}{k}\left(v_{0}+\frac{mg}{k}\right)e^{-\frac{k}{m}t} - \frac{mg}{k}t\right]^{t_{0}}_{0}\\
  &= \scalebox{.8}[1]{$\di -\frac{mg}{k} \left(v_{0}+\frac{mg}{k}\right) \left(\frac{mg}{kv_{0}+mg}-1\right) - \frac{mg}{k}\frac{m}{k}\log\frac{kv_{0}+mg}{mg}$}\\
  &=\frac{mv_{0}}{k} - \frac{m^2}{k^2} g \log\frac{kv_{0}+mg}{mg}\\
  &=\frac{m}{k}\left(v_{0}-\frac{mg}{k}\log\frac{kv_{0}+mg}{mg}\right) \numbering{(答)}
\end{align*}

\begin{center}
  \includegraphics[width=.7\linewidth]{/Users/majinkuu/Desktop/2025:07:12 2.jpg}
\end{center}

\item  速さ$\di\frac{mg}{k}$で鉛直下向きに等速直線運動$\numbering{(答)}$
\end{mondaiA}
\end{multicols*}
\end{document}