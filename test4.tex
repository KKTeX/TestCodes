%%%%%%%%%%%%%%%%%%%%%%%%%%%%%%%%%%%%%%%%%%%%%%%%%%%%%%%%%%%%%%%%%%%%%%%%%%%%%%%%%%%%%%%%%
\documentclass[luatex,fontsize=8pt,paper=b5,twoside,report]{jlreq}%今後はこっちメインで
\usepackage[top=16truemm,bottom=14truemm,left=12truemm,right=12truemm]{geometry}%(電子書籍はこっち)
% \usepackage[top=16truemm,bottom=14truemm,outer=15truemm,inner=18truemm]{geometry}%(紙の書籍はこっち)
%%%%%%%%%%%%%%%%%%%%%%%%%%%%%%%%%%%%%%%%%%%%%%%%%%%%%%%%%%%%%%%%%%%%%%%%%%%%%%%%%%%%%%%%%


%パッケージ関連%%%%%%%%%%%%%%%%%%%%%%%%%%%%%%%%%%%%%%%%%%%%%%%%%%%%%%%%%%%%%%%%%%%%%%%%%%
\usepackage{/Users/majinkuu/Documents/TeXDevelop/styles2/packages-v1.0}
\usepackage{/Users/majinkuu/Documents/TeXDevelop/styles2/colors-v1.0}
\usepackage{/Users/majinkuu/Documents/TeXDevelop/styles2/fonts-v1.3}
\usepackage{/Users/majinkuu/Documents/TeXDevelop/styles2/pre-KKawaguchi-v2.0}
\usepackage{/Users/majinkuu/Documents/TeXDevelop/styles2/stfile-of-KKwaguchi-v1.6}
\usepackage{/Users/majinkuu/Documents/TeXDevelop/styles2/sectioncustomize2}
% \usepackage[usetype1]{/Users/majinkuu/Documents/TeXDevelop/styles2/使わないstyファイル/myuline--}
% \usepackage{background}
%%%%%%%%%%%%%%%%%%%%%%%%%%%%%%%%%%%%%%%%%%%%%%%%%%%%%%%%%%%%%%%%%%%%%%%%%%%%%%%%%%%%%%%%%


\begin{document}
\begin{ptbs}{\mbox{Caution!!}}[\hannarifamily 縦書き環境で気をつけるべきこと]
\hannarifamily
\begin{enumerate}
  \item tikzにおいて原点が右上に、$x$軸正方向が↓に、$y$軸正方向が→になる\footnote{それによりテキストボックスもちゃんと$\pi/2$回転した上で描画されてくれるのでそのまま使える。}
  \item (\seihou{1}に連動し)\sou{\tbk vspace}, \sou{\tbk vskip}が紙面左右方向のスペースを制御する(逆もまた然り)
  \item \seihou{1}, \seihou{2}は\sou{fancyhdr}の描画には影響しない
  \item 通常の下線コマンド(\sou{lua-ul}の\sou{\tbk underLine}など)をそのまま使ってしまうと文字に被ってしまいかねない
  \item \sou{uline--, myuline--}で線を引くと2重になってしまうことがある
  \item \sou{twoside}指定時に冊子の開き方が逆になる
\end{enumerate}
\end{ptbs}

\begin{mydec3}{\TeX \hspace{1ex} Output}
  \vskip5mm\begin{center}
  \renewcommand{\arraystretch}{1.1} % 行間を少し広めに
  \begin{tabular}{|c||c|c|c|c|c|c|c|}
  \hline
  $x$       & $\cdots$ & $0$ & $\cdots$ & $1$ & $\cdots$ & $2$ & $\cdots$ \\
  \hline
  $f'(x)$   & $+$      &     & $+$      & $0$ & $-$      &     & $-$      \\
  \hline
  $f''(x)$  & $+$      & $0$ & $-$      &     & $-$      & $0$ & $+$      \\
  \hline
  $f(x)$    & \neRround &     & \neLround &     & \seLround &     & \seRround \\
  \hline
  \end{tabular}
  \end{center}\vskip5mm
\end{mydec3}
\end{document}